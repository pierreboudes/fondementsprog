% Created 2017-09-05 Tue 19:37
% Intended LaTeX compiler: pdflatex
\documentclass[11pt, a4paper]{article}
\usepackage[utf8]{inputenc}
\usepackage[T1]{fontenc}
\usepackage{graphicx}
\usepackage{grffile}
\usepackage{longtable}
\usepackage{wrapfig}
\usepackage{rotating}
\usepackage[normalem]{ulem}
\usepackage{amsmath}
\usepackage{textcomp}
\usepackage{amssymb}
\usepackage{capt-of}
\usepackage{hyperref}
\usepackage[margin=2cm]{geometry}
\usepackage[table]{xcolor}
\usepackage{xspace}
\usepackage{multicol}
\usepackage{bussproofs}
\usepackage{tikz}\usetikzlibrary{arrows,shapes,trees}
\renewcommand{\maketitle}{{\bigskip{\begin{center}\Large\textbf{Fondements de la programmation}\\[0.1cm] Exercices 8 lambda-calcul\end{center}}}\smallskip}
\usepackage{fancyhdr}
\usepackage[french, frenchb]{babel}
\author{Pierre Boudes}
\date{\today}
\title{Fondements de la programmation lambda-calcul}
\hypersetup{
 pdfauthor={Pierre Boudes},
 pdftitle={Fondements de la programmation lambda-calcul},
 pdfkeywords={},
 pdfsubject={},
 pdfcreator={Emacs 25.1.1 (Org mode 9.0.9)},
 pdflang={Frenchb}}
\begin{document}

\maketitle
\EnableBpAbbreviations
\pagestyle{fancyplain}
\fancyhf{}
\lhead{ \fancyplain{}{\raisebox{-1ex}{\includegraphics[scale=0.10]{../img/logoLipnNoir.pdf}} P. Boudes, P. Jacobé de Naurois, D. Mazza, V. Mogbil}}
\rhead{ \fancyplain{}{M1 informatique 2016-2017}}
\rfoot{ \fancyplain{}{\thepage}}
%\rfoot{ }
\newcounter{questioncount}
\setcounter{questioncount}{0}
\newcommand{\question}[1][]{\addtocounter{questioncount}{1}\paragraph{Question \Alph{questioncount}. #1}}
\renewcommand{\subsubsection}[1]{\question[#1.]}




\begin{multicols}{2}



Un \textbf{type simple} est soit un élément d'un ensemble de types atomiques, soit une flèche entre deux types :
\[
A := \alpha \mid A\implies B
\]
où \(\alpha\) est un type atomique (ensemble choisi au départ).

Un lambda-terme \(t\) est typable lorsqu'on peut lui donner un type (ici
parmi les types simples). Les lambda-termes typables sont définis par
induction sur la syntaxe.

Intuitivement cela se fait comme ceci (mais nous allons voir qu'il y a
un problème).
\begin{gather}
t := x^A \mid \overbrace{ \lambda x^A. t^B }^{A\implies B} \mid ( \overbrace{ u^{A\implies B} v^A) }^{B}
\label{intuit}
\end{gather}
%\vspace{-\baselineskip}
Une variable est toujours typable, et on peut lui donner
n'importe quel type. Si \(u\) est un lambda-terme typable auquel on peut
donner le type \(A\implies B\) et si \(v\) est un lambda-terme typable auquel
on peut donner le type \(A\), alors \((u\; v)\) est un lambda-terme
typable et on peut lui donner le type \(B\). Enfin si \(t\) est un
lambda-terme typable auquel on peut donner le type \(B\) et \(x\) est une
variable à laquelle \emph{on a donné} le type \(A\) en typant \(t\), alors
\(\lambda x. t\) est typable, de type \(A\implies B\). Le point important ici,
est que toutes les occurrences de la variable libre \(x\) doivent avoir
ce même type \(A\) à l'intérieur de \(t\). Mais comment l'exprimer
correctement ?

Cela se fait en introduisant les notions de contexte de typage et de
jugement de typage dans l'induction. Un contexte de typage, est un
tableau associatif (un dictionnaire), dont les clés sont des
variables et dont les valeurs sont des types simples. Un contexte de
typage associe donc un unique type simple à chaque variable d'un
ensemble fini donné de variables. On note traditionnellement \(x : A\),
l'association du type \(A\) à la variable \(x\), \(\Gamma\) un contexte de
typage et on utilise la virgule pour dénoter l'union disjointe de
contextes. Ainsi \(\Gamma, x : A\) signifie le contexte de typage formé
en prenant un contexte \(\Gamma\) non défini pour la clé \(x\) et en
l'étendant en donnant à la clé \(x\) la valeur \(A\).

Un jugement de typage \(\Gamma \vdash t: A\) permet d'associer un type
\(A\) à un terme \(t\) dans un contexte \(\Gamma\). On peut alors formaliser
l'induction précédente en disant qu'un jugement de typage est valide
s'il peut être dérivé grâce aux régles suivantes :
\begin{gather*}
  \AXC{}\RL{id}
\UIC{$\Gamma, x:A\vdash x:A$}
\DP\quad
\AXC{$\Gamma, x:A\vdash t:B$}\RL{abs.}
\UIC{$\Gamma \vdash \lambda x. t:A \implies B$}
\DP\\[0.5cm]
\AXC{$\Gamma\vdash u:A\implies B$}
\AXC{$\Gamma\vdash v:A$}\RL{app.}
\BIC{$\Gamma \vdash (u\; v): B$}
\DP
\end{gather*}
Le point important est que pour la règle d'application, \(u\) et \(v\) ont
été typés en utilisant un même contexte, c'est à dire en donnant le
même type aux variables de même nom. Si un jugement de typage est
valide, le contexte de typage contient nécessairement toutes les
variables libres du terme.

Ces règles définissent un système d'inférence similaire au calcul des
séquents. Une inférence de typage est un arbre dont chaque nœud est
un instance de règle, chaque feuille une instance d'axiome, de telle
sorte que chaque arête mette en relation un même séquent à chaque
extrémité.

Une autre façon pratique de voir le typage des lambda-termes est de ne
considérer que des lambda-termes dans lesquels les variables liées
sont choisies toutes différentes par \$\(\alpha\)\$-renommage (on ne peut
pas écrire \((\lambda x. x\; \lambda x. x)\), on écrit \((\lambda x. x\;
\lambda y. y)\)) et dans lesquels on a choisi pour chaque variable un
type unique (on annote les occurrences de variables dans le terme avec
leur type). Ce choix est l'équivalent du choix d'un contexte de typage
étendu aux variables liées. Si le lambda-terme est typable dans un tel contexte,
il a alors un type unique et on peut trouver son type en suivant la
structure syntaxique du terme comme dans \ref{intuit}.

Un lambda-terme pur peut ne correspondre à aucun terme typé comme il
peut correspondre à plusieurs termes typés (il peut n'y avoir aucune
façon de le décorer avec des types comme il peut y en avoir
plusieurs).


Les types sont conservés par \$\(\beta\)\$-conversion. (La substitutution est
faite en utilisant un terme de même type que la variable substituée).

Il existe des termes non typables qui se réduisent en des
termes typables.

Le calcul typé est fortement normalisant : n'importe quelle série de
réductions amène à une forme normale (et cette forme normale est unique
pour tous les chemins de réduction). Il n'est donc pas Turing complet.

\end{multicols}

\section{Exercices}
\label{sec:org5696ae6}

\subsection{Inférence de type}
\label{sec:org955c558}
Donner un type à chacun des termes suivants, lorsque c'est possible :
\begin{multicols}{2}

\begin{enumerate}
\item \(\lambda x. x\)
\item \(\lambda xf. (f\; (f\; (f\; x)))\)
\item \(\lambda xyz. ((x\; y)\; z)\)
\item \(((\lambda x. x)\; (\lambda x. x))\)
\item \((x\; y)\)
\item \((x\; x)\)
\item \(\lambda f.(\lambda x.(f\; (x\; x))) (\lambda x.(f\; (x\; x)))\)
\item Les entiers de Church et les opérations sur ces entiers.
\end{enumerate}

\end{multicols}


\subsection{Terme non typable}
\label{sec:org50383b0}
Donner un exemple de terme non typable qui se réduit en un terme
typable.

\subsection{Termes d'un type donné}
\label{sec:orga64521e}
Pour chacun des types suivants, donner un terme clos (sans variable libre) de ce type.
\begin{enumerate}
\item \(\alpha\implies\alpha\)
\item \((\alpha\implies\alpha)\implies(\alpha\implies\alpha)\)
\item \(\alpha\implies((\alpha\implies\beta)\implies\beta)\).
\end{enumerate}


\subsection{Prédécesseur (difficile)}
\label{sec:org8d1e0ce}

Trouver un \$\(\lambda\)\$-terme \(\operatorname{pred}\) qui calcule le
prédécesseur sur les entiers de Church, c'est à dire tel que appliqué
à un entier \(\overline{n}\) de Church se \$\(\beta\)\$-réduit en \(\overline{n -
1}\) si \(n > 0\) et en \(\lambda fx. x\) (c'est à dire \(\overline{0}\)) sinon.

\subsection{Combinateur de point fixe Y de Church}
\label{sec:org19639bd}
Soit :
\begin{gather*}
   Y = \lambda f.(\lambda x.(f\; (x\; x))) (\lambda x.(f\; (x\; x)))
  \end{gather*}
Dans le lambda-calcul pur, pour \(f\) terme quelconque, montrer que \((Y f) \equiv_{\beta} (f (Y f))\).

\subsection{Fonction récursive en lambda-calcul}
\label{sec:org7a3087d}
Définir la fonction factorielle en lambda-calcul pur (sur les entiers
de Church). Indication : cette fonction doit être le point fixe d'une
fonction se prenant elle-même en entrée et définie par cas sur son
deuxième argument (l'entier).
\end{document}