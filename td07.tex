% Created 2017-09-05 Tue 19:37
% Intended LaTeX compiler: pdflatex
\documentclass[11pt, a4paper]{article}
\usepackage[utf8]{inputenc}
\usepackage[T1]{fontenc}
\usepackage{graphicx}
\usepackage{grffile}
\usepackage{longtable}
\usepackage{wrapfig}
\usepackage{rotating}
\usepackage[normalem]{ulem}
\usepackage{amsmath}
\usepackage{textcomp}
\usepackage{amssymb}
\usepackage{capt-of}
\usepackage{hyperref}
\usepackage[margin=2cm]{geometry}
\usepackage[table]{xcolor}
\usepackage{xspace}
\usepackage{multicol}
\usepackage{bussproofs}
\usepackage{tikz}\usetikzlibrary{arrows,shapes,trees}
\renewcommand{\maketitle}{{\bigskip{\begin{center}\Large\textbf{Fondements de la programmation}\\[0.1cm] Exercices 7 lambda-calcul\end{center}}}\smallskip}
\usepackage{fancyhdr}
\usepackage[french, frenchb]{babel}
\author{P. Boudes}
\date{\today}
\title{Fondements de la programmation lambda-calcul}
\hypersetup{
 pdfauthor={P. Boudes},
 pdftitle={Fondements de la programmation lambda-calcul},
 pdfkeywords={},
 pdfsubject={},
 pdfcreator={Emacs 25.1.1 (Org mode 9.0.9)},
 pdflang={Frenchb}}
\begin{document}

\maketitle
\pagestyle{fancyplain}
\fancyhf{}
\lhead{ \fancyplain{}{\raisebox{-1ex}{\includegraphics[scale=0.10]{../img/logoLipnNoir.pdf}} P. Boudes, P. Jacobé de Naurois, D. Mazza, V. Mogbil}}
\rhead{ \fancyplain{}{M1 informatique 2016-2017}}
\rfoot{ \fancyplain{}{\thepage}}
%\rfoot{ }
\newcounter{questioncount}
\setcounter{questioncount}{0}
\newcommand{\question}[1][]{\addtocounter{questioncount}{1}\paragraph{Question \Alph{questioncount}. #1}}
\renewcommand{\subsubsection}[1]{\question[#1.]}




\begin{multicols}{2}

Un lambda-terme \(t\) est une expression syntaxique qui est soit
simplement une variable, soit l'abstraction d'un autre lambda-terme
selon une variable, soit l'application d'un terme à un autre :

\[
  t := x \mid \lambda x. t \mid (t\; t)
\]

\noindent où \(x\) est n'importe quel élément d'un ensemble de symboles
\(\mathcal{V}\) choisi une fois pour toute comme ensemble de variables.
On utilise habituellement les lettres de l'alphabet \(a\), \(b\), \ldots,
\(x\), \(y\), \(z\) éventuellement indexés par des entiers, \(x_{42}\), comme
ensemble de variables. Il est habituel de raccourcir les notations des
lambdas en notant \(\lambda x_1\ldots x_n. t\) au lieu de \(\lambda
x_1.\lambda x_2.\ldots \lambda x_n. t\). Par exemple on écrira \(\lambda
xy. x\) au lieu de \(\lambda x.\lambda y. x\). On doit parfois ajouter
des parenthèses autour des sous-expressions \(\lambda x. t\) pour éviter
une ambiguité. Par exemple, on écrit \(((\lambda x. x)\; y)\). Par
contre on peut omettre nombre de parenthèses d'application en prenant
comme convention que les expressions comme \(x\; y\; z\) sont
parenthésées à gauche: \(((x\; y)\; z)\).

L'ensemble des variables libres d'un lambda-terme est défini par
induction sur le terme :
\newcommand{\FV}[1]{\ensuremath{\operatorname{FV}(#1)}}
  \begin{align*}
    \FV x &= \{x\}\\
    \FV{\lambda x. t} &= \FV{t} - \{x\}\\
    \FV{(t\; u)} &= \FV{t} \cup \FV{u}
  \end{align*}
De même, l'ensemble des variables liées d'un lambda-terme est défini
par induction sur le terme :
\newcommand{\BV}[1]{\ensuremath{\operatorname{BV}(#1)}}
  \begin{align*}
    \BV x &= \emptyset \\
    \BV{\lambda x. t} &= \BV{t} \cup \{x\}\\
    \BV{(t\; u)} &= \BV{t} \cup \BV{u}
 \end{align*}
Remarques : ce sont les lambdas qui lient les variables; une variable
libre est simplement une variable qui n'est pas liée.


Lorsqu'on écrit un lambda-terme sous forme d'arbre syntaxique, les
\(\lambda x. \dots\) sont des nœuds unaires, les applications sont des nœuds
binaires, habituellement dénotés par le symbole @, et les variables
correspondent aux feuilles. On peut alors ajouter à cet arbre
syntaxique des arcs orientés. Pour chaque variable liée on dessine un
arc allant au lambda qui la lie (le premier lambda en remontant vers la racine).

Avec les arcs lieurs on obtient une arbre appelé arbre du
lambda-terme. Un exemple est donné figure 1.

%\begin{figure}[htbp]
\begin{wrapfigure}{r}{0.2\textwidth}
\vspace{-0.7cm}~\hspace{-0.5cm}
\begin{tikzpicture}[level distance=1cm]
  \node (lx) {$\lambda x$}
    child { node (ly) {$\lambda y$}
      child { node {@}
        child { node {@}
          child { node (x) {$x$} }
          child { node (y1) {$y$} }
        }
        child { node (y2) {$y$} }
      }
   };
  \path[->] (y1) edge [out=0, in = -20,  looseness = 1.2]  (ly);
  \path[->] (y2) edge [out=40, in = -40,  looseness = 0.8] (ly);
  \path[->] (x) edge [out=110, in = 250,  looseness = 1](lx);
  \end{tikzpicture}
\vspace{-0.5cm}
\label{fig:as}
\caption{arbre du terme $\lambda xy. ((x\; y)\; y)$}
\end{wrapfigure}
%\end{figure}

Deux termes dont les arbres sont égaux sauf éventuellement pour le nom
de leurs variables liées sont dits \$\(\alpha\)\$-équivalents. On peut
toujours renommer n'importe quelle variable liée d'un terme ou d'un
sous-terme pour obtenir un terme \$\(\alpha\)\$-équivalent.

Sur les arbres, la \textbf{substitution} d'une variable libre \(x\) par un terme
\(u\) dans un terme \(t\) consiste en remplacer dans l'arbre de \(t\) chaque
feuille \(x\) non liée par l'arbre de \(u\). On note \(t[x:=u]\) le
lambda-terme obtenu.

Si on n'y prend pas garde on risque la \emph{capture} de variables : des
variables qui étaients libres dans \(u\) se retrouvent liées dans
\(t[x:=u]\). On considéra uniquement la substitution sans capture de
variables. C'est à dire que l'opération de substitution n'a de sens qu'à
\$\(\alpha\)\$-renommage près. Toute capture d'une variable
libre de \(u\) par un lieur de \(t\) sera évitée par un renommage des
variables liées de \(t\) en choisissant des noms différents des
variables libres de \(u\).

Un terme de la forme \(((\lambda x. t)\; u)\) est appelé un \textbf{rédex}. Il
se \(\beta\) réduit en une étape en le terme \(t[x:=u]\). Un rédex peut
aussi apparaître dans un sous-terme (une sous-expression). L'idée est
celle de l'application d'une fonction à un argument : l'argument se
substitue à la variable de la fonction.

Un terme sans rédex est dit en forme normale.

Un terme peut avoir plusieurs rédex comme sous-termes. La
\$\(\beta\)\$-réduction peut alors s'appliquer à n'importe lequel des
sous-termes. Le choix du rédex à réduire peut éventuellement être déterminé
par une stratégie de réduction.

Partant d'un terme avec rédex si on arrive à une forme normale alors elle
est unique, quel que soit le chemin de réduction.

\end{multicols}

\section{Exercices}
\label{sec:org6aa5038}

\subsection{Variables libres, variables liées, alpha-équivalence}
\label{sec:orgc64374a}
\subsubsection{Variables libres et liées}
\label{sec:orgfa2af5d}
Donner l'ensemble des variables libres et l'ensemble des variables liées
des termes suivants :
\begin{multicols}{3}
\begin{enumerate}
\item \((x\; y)\)
\item \((x\; (\lambda y. y))\)
\item \(\lambda xyz. (x\; (y\; z))\)
\item \(((\lambda xyz. (x\; y))\; z)\)
\item \(((\lambda x. (x\; y))\; x)\)
\item \(((\lambda x. x)\;(\lambda x. x))\)
\end{enumerate}
\end{multicols}
\subsubsection{\$\(\alpha\)\$-équivalence}
\label{sec:org9fcfdc6}
Pour chacun des deux derniers termes de l'exercice précédent donner un terme qui lui est
\$\(\alpha\)\$-équivalent et utilise un maximum de noms de variables.
\subsection{Substitution, réductions}
\label{sec:orgf10ca7b}
\subsubsection{Substitutions}
\label{sec:org6050be9}
Effectuer les substitutions suivantes :
\begin{multicols}{3}
\begin{enumerate}
\item \(x[x:=y]\)
\item \((x\; y)[x:=\lambda z. z]\)
\item \((x\; x)[x:=\lambda z. z]\)
\item \((x\; x)[y:=\lambda z. z]\)
\item \(((\lambda x. (x\; y))\; x)[x:=z]\)
\item \(((\lambda x. (x\; y))\; y)[y:=z]\)
\end{enumerate}
\end{multicols}

\subsubsection{Réductions}
\label{sec:orgbb8a400}
Réduire et donner la forme normale des termes suivants :
\begin{multicols}{2}
\begin{enumerate}
\item \(((\lambda x. x)\;((\lambda x. x)\;(\lambda x. x)))\)
\item \(((\lambda x. (x\; x))\;(\lambda z. z))\)
\item \(((\lambda xy. y)\; f\; g)\)
\item \(((\lambda xy. (x\; (x\; (x\; y))))\; (\lambda x.z))\)
\item \(((\lambda x. (x\; x))\;(\lambda x. (x\; x)))\)
\item \(((\lambda f.(\lambda x.f (x\; x))\; (\lambda x.f (x\; x)))\; g)\)
\end{enumerate}
\end{multicols}
\subsection{Entiers de Church}
\label{sec:orga775498}
On se donne un ensemble de lambda-termes pour représenter les entiers
naturels. On représente le zéro par le terme \(\lambda fx. x\), le un
par le terme \(\lambda fx. (f\; x)\), le deux par le terme \(\lambda fx.
 (f\; (f\; x))\) et plus généralement l'entier \(n\) est représenté par
le terme noté \(\overline{n}\) : \[ \overline{n} := \lambda fx.
 (\underbrace{f\;\ldots\; (f}_{n}\; x)). \] Cette représentation est
appelée entiers de Church du nom du mathématicien Alonzo Church, qui
inventa le lambda-calcul et montra que toutes les fonctions
calculables sur les entiers peuvent être représentées par des
lambda-termes en utilisant la \$\(\beta\)\$-réduction pour effectuer le
calcul. Nous allons essayer de calculer un peu avec ses entiers.

\subsubsection{Successeur}
\label{sec:orgedd3c4e}
Donner un lambda-terme \(\operatorname{succ}\) qui représente l'opération
successeur sur les entiers de Church, c'est à dire tel que pour
n'importe quel entier naturel \(n\), \((\operatorname{succ}\; \overline{n})\) se
\$\(\beta\)\$-réduit en \(\overline{n + 1}\). Tester.

\subsubsection{Addition}
\label{sec:orgf779193}
Donner un lambda-terme \(\operatorname{add}\) qui représente l'addition de
deux entiers. C'est-à-dire tel que pour n'importe quels entiers \(n\) et
\(n\), \((\operatorname{add}\; \overline{n}\; \overline{m})\) se
\$\(\beta\)\$-réduit en \(\overline{n + m}\).

\subsubsection{Multiplication}
\label{sec:org67101b5}
Même question pour la multiplication \(\overline{mult}\).
\subsubsection{Test à zéro}
\label{sec:org9144ad7}
Donner un lambda-terme \(\operatorname{ifzero}\) qui, lorsqu'il est appliqué à un entier
\(\overline{n}\) et deux termes \(t_1\) et \(t_2\), se \$\(\beta\)\$-réduit en \(t_1\)
si \(n = 0\) et en \(t_2\) sinon.
\end{document}