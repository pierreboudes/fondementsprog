% Created 2017-09-05 Tue 19:35
% Intended LaTeX compiler: pdflatex
\documentclass[11pt, a4paper]{article}
\usepackage[utf8]{inputenc}
\usepackage[T1]{fontenc}
\usepackage{graphicx}
\usepackage{grffile}
\usepackage{longtable}
\usepackage{wrapfig}
\usepackage{rotating}
\usepackage[normalem]{ulem}
\usepackage{amsmath}
\usepackage{textcomp}
\usepackage{amssymb}
\usepackage{capt-of}
\usepackage{hyperref}
\usepackage[margin=2cm]{geometry}
\usepackage[table]{xcolor}
\usepackage{xspace}
\usepackage{multicol}
\usepackage{bussproofs}
\usepackage{tikz}\usetikzlibrary{arrows,shapes,trees}
\renewcommand{\maketitle}{{\bigskip{\begin{center}\Large\textbf{Fondements de la programmation}\\[0.1cm] Exercices 9 et 10 lambda-calcul\end{center}}}\smallskip}
\usepackage{fancyhdr}
\usepackage[french, frenchb]{babel}
\author{Pierre Boudes}
\date{\today}
\title{Fondements de la programmation lambda-calcul}
\hypersetup{
 pdfauthor={Pierre Boudes},
 pdftitle={Fondements de la programmation lambda-calcul},
 pdfkeywords={},
 pdfsubject={},
 pdfcreator={Emacs 25.1.1 (Org mode 9.0.9)},
 pdflang={Frenchb}}
\begin{document}

\maketitle
\EnableBpAbbreviations
\pagestyle{fancyplain}
\fancyhf{}
\lhead{ \fancyplain{}{\raisebox{-1ex}{\includegraphics[scale=0.10]{../img/logoLipnNoir.pdf}} P. Boudes, P. Jacobé de Naurois, D. Mazza, V. Mogbil}}
\rhead{ \fancyplain{}{M1 informatique 2016-2017}}
\rfoot{ \fancyplain{}{\thepage}}
%\rfoot{ }
\newcounter{questioncount}
\setcounter{questioncount}{0}
\newcommand{\question}[1][]{\addtocounter{questioncount}{1}\paragraph{Question \Alph{questioncount}. #1}}
\renewcommand{\subsubsection}[1]{\question[#1]}
\newcommand{\tabDP}{\shortstack{\smallskip\\\DisplayProof\smallskip}}
\newcommand{\eqdef}{\mathrel{\shortstack{\scriptsize\text{def}\\=}}}
\newcommand{\fix}{\mathop{\texttt{fix}}}



\begin{multicols}{2}

Stratégie d'évaluation : \textbf{appel par valeur.} Nous allons maintenant
restreindre la \(\beta\)-réduction de façon à ce qu'elle corresponde à la façon dont
les programmes sont évalués dans la plupart des langages de
programmation. On distingue un ensemble particulier de termes appelés
valeurs au delà desquels on ne (\(\beta\)-)réduit plus. En l'absence
d'extensions du lambda-calcul, qui viendront après, les valeurs sont
tous les termes qui correspondent à une abstraction :
\[
v := \lambda x. t
\]
Intuitivement, on ne réduit pas dans le corps des fonctions tant
qu'elles ne sont pas complètement appliquées.

On se donne
ensuite des règles d'évaluation qui déterminent exactement l'ordre dans
lequel il faut faire les réductions dans un terme.
\begin{gather*}
\AXC{}
\UIC{$(\lambda x. t_1\; v_2) \to t_1[x:=v_2]$}\tabDP\\
  \AXC{$t_1 \to t'_1$}
  \UIC{$(t_1\; t_2)\to (t'_1\; t_2)$}
\tabDP\quad
  \AXC{$t_2 \to t'_2$}
  \UIC{$(v_1\; t_2)\to (v_1\; t'_2)$}
\tabDP
\end{gather*}

Le point important dans cet ensemble de règles est qu'il n'y a pas de
règle qui permette de réduire sous une abstraction.

Nous reviendrons plus en détails sur les différentes stratégies
d'évaluation possibles.

\textbf{Unicité du typage.} Nous utilisons désormais une version du lambda-calcul
simplement typé où chaque variable liée par un lambda est annotée par
un type, comme ceci : \(\lambda x[T]. t\), de telle sorte que \(T\) soit
le type associé à la variable \(x\) au moment du jugement de typage :

\begin{gather*}
  \AXC{$\Gamma, x:A\vdash t:B$}\RL{abs.}
  \UIC{$\Gamma \vdash \lambda x[A]. t:A \Rightarrow B$}
\DP
\end{gather*}
De cette façon, étant donné un contexte de typage \(\Gamma\), et un
terme \(t\) ainsi annoté il existe au plus un type \(T\) tel que
\(\Gamma\vdash t: T\).

Nous allons maintenant
donner des \textbf{extensions du lambda-calcul} \textbf{simplement typé} de façon à en
faire un langage de programmation, Turing-complet et ayant les
caractéristiques d'un langage de type ML (les langages ML comme SML,
CAML sont inspirés des extensions du lambda-calcul et en particulier
du langage PCF, \emph{programming} \emph{computable} \emph{functions}).

\textbf{Unit}. On ajoute une valeur
\texttt{unit}  de type, \texttt{Unit}.
\begin{align*}
  t &:= \ldots \mid \texttt{unit}\\
  v &:= \ldots \mid \texttt{unit}\\
  T &:= \ldots \mid \texttt{Unit}
\end{align*}  \begin{gather*}
  \AXC{}
  \UIC{$\Gamma \vdash   \texttt{unit} : \texttt{Unit}$}\DP
\end{gather*}
Ceci nous permet d'introduire une \emph{forme dérivée}:
\[
t_1;t_2 \eqdef (\lambda x[\texttt{Unit}]. t_2\;\; t_1)\quad (x\notin FV(t_2)).
\]
Cette mise en séquence, \(t_1; t_2\) a pour effet d'évaluer
\(t_1\), de jeter son résultat, puis d'évaluer \(t_2\). Les règles
d'évaluation et de typage induites sur la forme sont :
\begin{gather*}
\AXC{$t_1\to t'_1$}
\UIC{$t_1; t_2\to t'_1; t_2$}
\tabDP\quad
\AXC{}
\UIC{$\texttt{unit}; t_2 \to t_2$}\tabDP
\\
\AXC{$\Gamma \vdash t_1 : \texttt{Unit}$}
\AXC{$\Gamma \vdash t_2: T_2$}
\BIC{$\Gamma \vdash t_1; t_2: T_2$}\tabDP
\end{gather*}

\textbf{Let in}.  On peut également se donner la forme syntaxique \texttt{let in}
  \begin{gather*}
   t := \ldots \mid \texttt{let }x = t\texttt{ in } t\\
  \AXC{}
  \UIC{$\texttt{let }x = v_1\texttt{ in } t_2 \to t_2[x := v_1]$}\tabDP\\
  \AXC{$t_1 \to t'_1$}
  \UIC{$\texttt{let }x = t_1\texttt{ in } t_2 \to \texttt{let }x = t'_1\texttt{ in } t_2 $}\tabDP\\
\AXC{$\Gamma \vdash t_1: T_1$}
\AXC{$\Gamma, x : T_1 \vdash t_2: T_2$}
\BIC{\Gamma\vdash \texttt{let }x = t_1\texttt{ in } t_2 :  T_2$}
\tabDP
  \end{gather*}

Au premier abord, cette forme \(\texttt{let }x = t_1\texttt{ in } t_2\)
est juste du sucre syntaxique pour \((\lambda x[T_1]. t_2\; t_1)\). Pour
\emph{désugariser} il faudra toutefois trouver l'annotation de type pour
\(\lambda x\), donc faire l'inférence de type avant.

\textbf{Combinateur de point fixe}. Pour chaque type \(T_1\) on se donne
 un  opérateur sur les termes \texttt{fix} tel que :
  \begin{gather*}
    t := \ldots \mid (\fix\; t)\\
  \AXC{}
  \UIC{$(\fix \;\lambda x[T_1]. t_2) \to t_2 [x:=(\fix \;\lambda x[T_1]. t_2)]$}\tabDP\\
  \AXC{$t_1 \to t'_1$}
  \UIC{$\fix\; t_1 \to \fix\; t'_1$}\tabDP\\
\AXC{$\Gamma \vdash t_1: T_1\to T_1$}
\UIC{$\Gamma \vdash (\fix \; t_1): T_1$}
\tabDP  \end{gather*}

Forme dérivée:
  \begin{align*}
 &\texttt{letrec } x =  t_1 \texttt{ in }   t_2 \\
&\eqdef
  \texttt{let } x = (\fix \;\lambda x[T_1]. t_1) \texttt{ in }   t_2
  \end{align*}


On peut construire des types de données composées, le plus simple
d'entre eux étant les \textbf{paires} :
\begin{align*}
  t &:= \ldots \mid \{t, t\} \mid t.1 \mid t.2\\
  v &:= \ldots \mid \{t, t\}\\
  T &:= \ldots \mid T_1\times T_2
\end{align*}
Règles de réduction des paires :
  \begin{gather*}
  \AXC{$t\to t'$}
\LL{$i = 1, 2$}
  \UIC{$t.i\to t'.i$}
  \tabDP\quad
  \AXC{}
\RL{$i = 1, 2$}
  \UIC{$\{v_1, v_2\}.i \to v_i$}\tabDP
  \\
  \AXC{$t_1\to t'_1$}
  \UIC{$\{t_1, t_2\}\to \{t'_1, t_2\}$}
  \tabDP\quad
  \AXC{$t_2\to t'_2$}
  \UIC{$\{v_1, t_2\}\to \{v_1, t'_2\}$}
  \tabDP
  \end{gather*}
Règles de typage des paires :
\begin{gather*}
  \AXC{$\Gamma \vdash t_1: T_1$}
  \AXC{$\Gamma \vdash t_2: T_2$}
  \BIC{$\Gamma \vdash \{t_1, t_2\}: T_1 \times T_2$}\DP
\quad
\AXC{$\Gamma \vdash t: T_1\times T_2$}
\UIC{$\Gamma \vdash t.i : T_i$}
\LL{$i = 1, 2$}
\DP
\end{gather*}

\end{multicols}

\section{Exercices}
\label{sec:org3aeb5f3}

\subsubsection{Typage.}
\label{sec:org0287648}
Typer les termes suivants en utilisant les extensions appropriées du
lambda-calcul.
\begin{enumerate}
\item \(\texttt{let } f = \lambda xy. x \texttt{ in } \lambda abc. (f\;
   (f\; a\; b)\; c)\).
\item \((\lambda xy.x \; \texttt{unit} \; (\lambda z. z)); \lambda xy. y\)
\item \((\fix \; \lambda x. x)\)
\item \((\fix \; \lambda xy. x)\)
\item \(\texttt{letrec } f = \lambda ghx. (g \; x\; (f\; g\; h\; (h\; x))
   \; (h\; x))\texttt{ in } (f\; (\lambda xyz. z)\; (\lambda z. z))\)
\item \(\lambda x. x.1\)
\item \(\lambda xy. \{x, y\}\)
\end{enumerate}

\subsection{Letrec}
\label{sec:org4ad5f32}
\subsubsection{Question de partiel.}
\label{sec:orge2a9fe2}
Dans un \texttt{letrec} qu'est-ce qui s'apparente à du sucre syntaxique (ou une
forme dérivée) et qu'est ce qui est une véritable extension du
lambda-calcul simplement typé ?
\subsubsection{Fonctions récursives.}
\label{sec:org86d1dcc}
Définir factorielle à l'aide d'un \texttt{letrec} puis supprimer le sucre
syntaxique. De même pour la fonction de Fibonacci.

\subsubsection{Récursion.}
\label{sec:orgc8dafbd}
La plupart du temps un \texttt{letrec} sert à définir une fonction, c'est à
dire un terme de type \(T_1\Rightarrow T_2\), pour un certain \(T_1\) et un
certain \(T_2\). Pouvez-vous trouver une expression définie
récursivement et qui soit d'un type autre qu'une flèche ?

\subsection{Preuves par induction sur les jugements de typage}
\label{sec:org3f9171e}
\subsubsection{Affaiblissement}
\label{sec:org8bb73ba}
(importation dans d'autres contextes).
Prouver que, en lambda-calcul simplement typé, si \(\Gamma\vdash t: T\)
alors pour n'importe quel \(x\not\in FV(t)\), \(\Gamma, x:T'\vdash t: T\).

En déduire que si \(\Gamma\vdash t: T\)
alors \(\Gamma,
x_1:T_1, \ldots, x_n:T_n\vdash t: T\),  pour n'importe quels
\(x_1,\ldots,x_n\not\in FV(t)\).

\subsubsection{Substitution}
\label{sec:orgbd1f1f1}
(modularité et liaison.)
Prouver que si \(\Gamma, x : T' \vdash t : T\) et \(\Gamma \vdash t' :
T'\), alors \(\Gamma \vdash t[x:=t'] : T\).
\end{document}