% Created 2017-09-05 Tue 19:29
% Intended LaTeX compiler: pdflatex
\documentclass[11pt, a4paper]{article}
\usepackage[utf8]{inputenc}
\usepackage[T1]{fontenc}
\usepackage{graphicx}
\usepackage{grffile}
\usepackage{longtable}
\usepackage{wrapfig}
\usepackage{rotating}
\usepackage[normalem]{ulem}
\usepackage{amsmath}
\usepackage{textcomp}
\usepackage{amssymb}
\usepackage{capt-of}
\usepackage{hyperref}
\usepackage[margin=1.8cm]{geometry}
\usepackage[table]{xcolor}
\usepackage{xspace}
\usepackage{multicol}
\usepackage{bussproofs}
\usepackage{tikz}\usetikzlibrary{arrows,shapes,trees}
\renewcommand{\maketitle}{{\bigskip{\begin{center}\Large\textbf{Fondements de la programmation}\\[0.1cm] Exercices 11 : extensions, type safety, machine de Krivine\end{center}}}\smallskip}
\usepackage{fancyhdr}
\usepackage[french, frenchb]{babel}
\author{Pierre Boudes}
\date{\today}
\title{Fondements de la programmation lambda-calcul}
\hypersetup{
 pdfauthor={Pierre Boudes},
 pdftitle={Fondements de la programmation lambda-calcul},
 pdfkeywords={},
 pdfsubject={},
 pdfcreator={Emacs 25.1.1 (Org mode 9.0.9)},
 pdflang={Frenchb}}
\begin{document}

\maketitle
\EnableBpAbbreviations
\pagestyle{fancyplain}
\fancyhf{}
\lhead{ \fancyplain{}{\raisebox{-1ex}{\includegraphics[scale=0.10]{../img/logoLipnNoir.pdf}} P. Boudes, P. Jacobé de Naurois, D. Mazza, V. Mogbil}}
\rhead{ \fancyplain{}{M1 informatique 2016-2017}}
\rfoot{ \fancyplain{}{\thepage}}
%\rfoot{ }
\newcounter{questioncount}
\setcounter{questioncount}{0}
\newcommand{\question}[1][]{\addtocounter{questioncount}{1}\paragraph{Question \Alph{questioncount}. #1}}
\renewcommand{\subsection}[1]{\question[#1]}
\newcommand{\tabDP}{\shortstack{\smallskip\\\DisplayProof\smallskip}}
\newcommand{\eqdef}{\mathrel{\shortstack{\scriptsize\text{def}\\=}}}
\newcommand{\fix}{\mathop{\texttt{fix}}}





\begin{multicols}{2}





On se donne un ensemble \(L\) de mots

ou étiquettes. Un \textbf{type somme} est la donnée d'un dictionnaire fini
dont les clés sont des mots et dont les valeurs sont des types:
\(\{\text{mot}_1: T_1, \ldots, \text{mot}_n: T_n\}\). On étend les types
avec les types sommes et les termes avec le pattern-matching.

\begin{align*}
T := \ldots &\mid \{\text{mot}_1: T_1, \ldots, \text{mot}_n: T_n\}\\
t := \ldots &\mid \texttt{match } t \texttt{ with } \\
&\quad\texttt{ case } \text{mot}_1\; x_1 \Rightarrow t_1\\
&\qquad\vdots\\
&\quad\texttt{ case } \text{mot}_n\; x_n\Rightarrow t_n\\
&\mid \text{mot } t\texttt{ as }T\\
v := \ldots &\mid \text{mot } t\texttt{ as }T
\end{align*}

Remarque : sans ce « \(\texttt{as }T\) » on perdrait l'unicité du type
associé à un terme (on aurait par exemple \(\text{Vrai}\;\texttt{unit}\) de type
\(\{\text{Vrai}: \texttt{Unit}\}\) mais aussi de type \(\{\text{Vrai}:
\texttt{Unit}, \text{Faux}: \texttt{Unit}\}\), ou de type \(\{\text{Vrai}:
\texttt{Unit}, \text{Saispas}: \texttt{Unit}\}\) etc.).

Règles de réduction.

\begin{gather*}
\AXC{$t\to t'$}
\UIC{$\texttt{match }t\texttt{ with}\ldots\to\texttt{match }t'\texttt{ with}\ldots$}
\tabDP\\
\AXC{$t\to t'$}
\UIC{$\text{mot}\; t\texttt{ as } T\to \text{mot } t'\texttt{ as } T$}
\tabDP\\
\AXC{}
\UIC{\shortstack[l]{
$\texttt{match } \text{m}_i\; t\texttt{ as } T  \texttt{ with }$ \\
$\qquad\quad\texttt{ case } \text{mot}_1\; x_1 \Rightarrow t_1$\\
$\qquad\qquad\vdots$\\
$\qquad\quad\texttt{ case } \text{mot}_n\; x_n\Rightarrow t_n$\\
$\to {t_i[x_i:=t]}$
}
}
\tabDP
\end{gather*}

Règles de typage.
\begin{gather*}
\AXC{$\Gamma\vdash t: \{\ldots, m_i: T_i, \ldots\}$}
\AXC{$\Gamma, x_i : T_i\vdash t_i: T \quad(\forall i)$}
\BIC{\shortstack[l]{
$\Gamma\vdash\texttt{match } t\texttt{ as } \{\ldots, m_i: T_i,\ldots\}  \texttt{ with }$ \\
$\qquad\ldots\texttt{ case } m_i\; x_i \Rightarrow t_i\ldots\quad : T$
}}
\tabDP\\
\AXC{$\Gamma\vdash t_i: T_i$}
\UIC{$\Gamma\vdash  \text{m}_i\; t_i\texttt{ as } \{\ldots, m_i: T_i, \ldots\}: \{\ldots, m_i: T_i, \ldots\}$}
\tabDP
\end{gather*}


\textbf{Théorème de préservation.} Si \(\Gamma \vdash t: T\) et \(t\to t'\),
alors \(\Gamma\vdash t': T\). Ceci est valable pour le lambda-calcul
simplement typé comme pour les extensions présentées ici. La
réciproque est en générale fausse.

\textbf{Théorème de progression.} Si \(\vdash t: T\) (notez le contexte vide),
alors soit \(t\) est une valeur, soit il existe \(t'\) tel que \(t\to t'\).

\textbf{Type safety}. Un langage est dit sûr au niveau du typage (\emph{type}
\emph{safe}) lorsqu'il possède les deux propriétés précédentes
(préservation et progression), éventuellement en élargissant la
propriété de progression de façon à permettre au calcul de terminer
sur des erreurs (des exceptions) aussi bien que sur des valeurs. Notez
que la progression n'implique pas la terminaison.


\textbf{Théorème de normalisation forte.} En lambda-calcul simplement typé
la \$\(\beta\)\$-réduction termine toujours (sans avoir besoin d'utiliser
une stratégie particulière).

\newcommand{\cons}{\mathrel{::}}

La \textbf{machine abstraite de Krivine (KAM)} implémente l'appel par nom en
\$\(\lambda\)\$-calcul. On définit 4 notions :
\begin{description}
\item[{terme}] \(t := x\mid (t\; t)\mid \lambda x. t\)
\item[{environnement}] \(e := \emptyset \mid e, x:c\) (un dictionnaire
associant des clôtures à des variables)
\item[{clôture}] \(c := (t, e)\)
\item[{pile}] \(\pi := \varepsilon \mid c\cons\pi\)
\end{description}
L'état de la machine est représenté par un triplet fait d'un terme
d'un environnement et d'une pile. On se donne un terme initial (un
programme) avec un environnement et une pile vide. Une étape de
réduction fait passer d'un état à l'état suivant en appliquant trois
règles :

\newcommand{\kpush}[3]{\RightLabel{push}\UIC{\ensuremath{#1\qquad #2\qquad #3}}}
\newcommand{\kpop}[3]{\RightLabel{pop}\UIC{\ensuremath{#1\qquad #2\qquad #3}}}
\newcommand{\kderef}[3]{\RightLabel{deref}\UIC{\ensuremath{#1\qquad
#2\qquad #3}}}

\begin{gather*}
\AXC{$(t\; u)\qquad e\qquad \pi$}\kpush{t}{e}{(u, e)\cons\pi}
\tabDP
\\
\AXC{$\lambda x. t\qquad e\qquad c:\pi$}\kpop{t}{e, x\cons c}{\pi}
\tabDP
\\
\AXC{$x\qquad e,x:(t, e')\qquad \pi$}\kderef{t}{e'}{\pi}
\tabDP
\end{gather*}

Par exemple, l'évaluation du terme \(((\lambda x. x)\; y)\) sur la KAM par nom se
déroule ainsi:

\begin{gather*}
\AXC{$((\lambda x. x)\; y) \qquad \emptyset\qquad \varepsilon$}
\kpush{\lambda x. x}{\emptyset}{(y, \emptyset)}
\kpop{x}{x:(y, \emptyset)}{\varepsilon}
\kderef{y}{\emptyset}{\varepsilon}
\DP
\end{gather*}


Le résultat est \(y\) (plus aucune règle ne s'applique).

La \textbf{KAM par valeur} est définie en modifiant
les piles de façon à ce qu'elles contiennent deux sortes d'éléments
(\emph{fonctions} et \emph{arguments}) :
\begin{description}
\item[{pile}] \(\pi = \varepsilon \mid F c \cons \pi \mid A c  \cons \pi\)
\end{description}

\newcommand{\kswap}[3]{\RightLabel{swap}\UIC{\ensuremath{#1\qquad #2\qquad #3}}}

Les règles sont, par priorités décroissantes:

\begin{gather*}
\AXC{$(t\; u)\qquad e\qquad \pi$}\kpush{t}{e}{A(u, e)\cons\pi}
\tabDP
\\
\AXC{$\lambda x. t\qquad e\qquad A(u, e')\cons\pi$}\kswap{u}{e'}{F(\lambda x.t, e)\cons\pi}
\tabDP
\\
\AXC{$x\qquad e,x=(t, e')\qquad \pi$}\kderef{t}{e'}{\pi}
\tabDP
\\
\AXC{$v\qquad e\qquad F(\lambda x. t, e')\cns\pi$}\kpop{t}{e',x=:(v, e)}{\pi}
\tabDP
\end{gather*}

Dans cette dernière règle, \(v\) désigne une valeur c'est à dire ici un
lambda-terme qui n'est pas une application (donc une variable ou un lambda).

Par exemple, l'évaluation du terme \(((\lambda x. x)\; y)\) sur la KAM par valeur se
déroule ainsi:
$$
\AXC{$((\lambda x. x)\; y) \qquad \emptyset\qquad \varepsilon$}
\kpush{\lambda x. x}{\emptyset}{A(y, \emptyset)}
\kswap{y}{\emptyset}{F(\lambda x. x, \emptyset)}
\kpop{x}{x=(y, \emptyset)}{\varepsilon}
\kderef{y}{\emptyset}{\varepsilon}
\DP
$$



\section{Exercices}
\label{sec:org33e1c23}
\subsection{Préservation.}
\label{sec:org4fdcbe1}
Prouver le théorème de préservation pour le \$\(\lambda\)\$-calcul
simplement typé.

\subsection{Normalisation forte.}
\label{sec:orge0c3444}
Pourquoi ne prouve t'on pas le théorème de normalisation forte du
lambda-calcul simplement typé par induction sur les termes ?


\subsection{Morris.}
\label{sec:orgeaea601}
En C que vaut Morris(1, 0) ? Citer un langage dans lequel cette
fonction s'évaluerait différemment.

\begin{verbatim}
int Morris(int a, int b)
{
  if (a == 0) {
    return 1;
  }
  else {
    return Morris(a - 1, Morris(a, b));
  }
}
\end{verbatim}


\subsection{Scala.}
\label{sec:orgea0f8a3}
En Scala, \texttt{val} est l'équivalent du \texttt{let} de Caml, il
permet d'introduire une variable immutable dont la valeur est obtenue en
appel par valeur, \texttt{var} permet d'introduire une variable
mutable dont la valeur est également fixée en appel par valeur,
\texttt{def} permet d'introduire une variable immutable en appel par
nom et \texttt{lazy val} est une variante de \texttt{val} qui est
évaluée paresseusement, en appel par nécessité. Un bloc d'instructions
(entre accolades) s'évalue comme la dernière valeur du bloc. Sauriez
vous dire quelle ligne de code provoque quel affichage dans l'extrait
de programme Scala suivant ?

\begin{verbatim}
val x = {println("x"); 1}
var y = {println("y"); 2}
def z = {println("z"); 3}
lazy val w = {println("w"); 4}
x + x
y + y
z + z
w + w
\end{verbatim}


\subsection{KAM par nom.}
\label{sec:org72ea082}
Donner l'exécution de la KAM par nom sur le terme \((((\lambda xy. x)\; z)\;
z')\).


\subsection{KAM par valeur.}
\label{sec:org06faa17}
Donner l'exécution de la KAM par valeur sur le terme \((((\lambda xy. x)\; z)\;
z')\).


\subsection{KAM (partiel 2015).}
\label{sec:org1e4f6e0}
On veut comparer le temps d'exécution de la KAM par nom et de la
KAM par valeur. Pour mesurer ce temps on compte le nombre de règles appliquées
dans chaque exécution mais \textbf{sans compter} les applications la règle
\emph{swap}. Ainsi les exécutions des deux machines sur le terme \(((\lambda x.
x)\; y)\) prennent autant de temps (3 règles hors règle \emph{swap}).
\begin{enumerate}
\item Est-ce encore le cas sur le terme \((((\lambda xy. x)\; z)\; z')\) ?
\item Les deux machines ont-elles toujours les mêmes temps d'exécution
ou bien pouvez-vous trouver un terme pour
lequel la KAM par nom est plus rapide et/ou un terme pour lequel la
KAM par valeur est plus rapide ? Justifier par un raisonnement ou
en donnant des exemples de termes et leurs exécutions.
\end{enumerate}
\end{document}